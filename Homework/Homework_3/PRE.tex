\documentclass[letterpaper]{article} 
\usepackage[utf8]{inputenc}
\usepackage[T1]{fontenc}
\usepackage{amsmath}
\usepackage{amsfonts}
\usepackage{amssymb}
\usepackage{array}
\usepackage{booktabs}
\usepackage{hyperref}
\usepackage[version=4]{mhchem}
\usepackage{stmaryrd}
\DeclareMathOperator{\Var}{Var}
\usepackage[dvipsnames]{xcolor}
\colorlet{LightRubineRed}{RubineRed!70}
\colorlet{Mycolor1}{green!10!orange}
\definecolor{Mycolor2}{HTML}{00F9DE}
\usepackage{graphicx}
\usepackage{amsmath}
\usepackage{graphicx}
\usepackage{capt-of}
\usepackage{lipsum}
\usepackage{fancyvrb}
\usepackage{tabularx}
\usepackage{listings}
\usepackage[export]{adjustbox}
\graphicspath{ {./images/} }
\usepackage[utf8]{inputenc}
\usepackage[english]{babel}
\usepackage{float}
\usepackage{lipsum}
\usepackage{graphicx}
\usepackage{float}
\usepackage[margin=0.7in]{geometry}
\usepackage{amsmath}
\usepackage{graphicx}
\usepackage{capt-of}
\usepackage{tcolorbox}
\usepackage{lipsum}
\usepackage{graphicx}
\usepackage{float}
\usepackage{listings}
\usepackage{hyperref} 
\usepackage{xcolor} % For custom colors
\lstset{
	language=Python,                % Choose the language (e.g., Python, C, R)
	basicstyle=\ttfamily\small, % Font size and type
	keywordstyle=\color{blue},  % Keywords color
	commentstyle=\color{gray},  % Comments color
	stringstyle=\color{red},    % String color
	numbers=left,               % Line numbers
	numberstyle=\tiny\color{gray}, % Line number style
	stepnumber=1,               % Numbering step
	breaklines=true,            % Auto line break
	backgroundcolor=\color{black!5}, % Light gray background
	frame=single,               % Frame around the code
}
\usepackage{float}
\usepackage[]{amsthm} %lets us use \begin{proof}
\usepackage[]{amssymb} %gives us the character \varnothing

	\title{Homework 4, STAT 5504}
	\author{Zongyi Liu}
	\date{Mon, Dec 8, 2025}
	\begin{document}
		\maketitle
		
		\section{11.1}
		\subsection{Question 1 (11.1.2)}
		
		Show that the value of $\hat\beta_1$ in Eq.\ (11.1.1) can be rewritten in
		each of the following three forms:
		\[
		\text{(a)}\quad 
		\hat\beta_1 = 
		\frac{\displaystyle\sum_{i=1}^n x_i y_i - n\bar x \bar y}
		{\displaystyle\sum_{i=1}^n x_i^2 - n\bar x^{\,2}},
		\qquad
		\text{(b)}\quad
		\hat\beta_1 =
		\frac{\displaystyle\sum_{i=1}^n (x_i-\bar x)y_i}
		{\displaystyle\sum_{i=1}^n (x_i-\bar x)^2},
		\qquad
		\text{(c)}\quad
		\hat\beta_1 =
		\frac{\displaystyle\sum_{i=1}^n x_i (y_i-\bar y)}
		{\displaystyle\sum_{i=1}^n (x_i-\bar x)^2}.
		\]
		
		\textbf{Answer}
		
		
		\clearpage
		
		\subsection{Question 2 (11.1.4)}
		
		For $i = 1,\ldots,n$, let $\hat{y}_i = \beta_0 + \beta_1 x_i$. Show that $\hat{\beta}_0$
		and $\hat{\beta}_1$, as given by Eq.\ (11.1.1), are the unique values of $\beta_0$
		and $\beta_1$ such that
		\[
		\sum_{i=1}^n (y_i - \hat{y}_i) = 0
		\qquad\text{and}\qquad
		\sum_{i=1}^n x_i (y_i - \hat{y}_i) = 0.
		\]
		
		\textbf{Answer}
		
		
		\clearpage
		
		\section{11.2}
		\subsection{Question 3 (11.2.6)}
		
		Show that in a problem of simple linear regression, the estimators 
		$\hat{\beta}_0$ and $\hat{\beta}_1$ will be independent if $\bar{x} = 0$.
		
		
		\textbf{Answer}
		
		
		\clearpage
		
		\subsection{Question 4 (11.2.12)}
		Consider a problem of simple linear regression in which the durability $Y$
		of a certain type of alloy is to be related to the temperature $X$ at which it
		was produced. Suppose that the eight pairs of observed values given in
		Table~11.3 are obtained. Determine the values of the M.L.E.'s 
		$\hat{\beta}_0$, $\hat{\beta}_1$, and $\hat{\sigma}^2$, and also the values of
		$\operatorname{Var}(\hat{\beta}_0)$ and $\operatorname{Var}(\hat{\beta}_1)$.
		
		
		\textbf{Answer}
		
		
		\clearpage
		
		\subsection{Question 5 (11.2.14)}
		Consider again the conditions of Exercise~12, and suppose that it is 
		desired to estimate the value of $\theta = 5 - 4\beta_0 + \beta_1$.  Find an
		unbiased estimator $\hat{\theta}$ of $\theta$.  Determine the value of 
		$\hat{\theta}$ and the M.S.E.\ of $\hat{\theta}$.
		
		
		\textbf{Answer}
		
		
		\clearpage
		
		\section{11.3}
		\subsection{Question 6 (11.3.2)}
		
		For the data presented in Table~11.9, test at the level of significance
		$0.05$ the hypothesis that the regression line passes through the origin in the
		$xy$-plane.
		
		\begin{table}[h!]

			\centering
			\caption{Data for Exercise }
			\begin{tabular}{c c c  c c c}
				\hline
				$i$ & $x_i$ & $y_i$ & $i$ & $x_i$ & $y_i$ \\
				\hline
				1 & 0.3 & 0.4 & 6 & 1.0 & 0.8 \\
				2 & 1.4 & 0.9 & 7 & 2.0 & 0.7 \\
				3 & 1.0 & 0.4 & 8 & -1.0 & -0.4 \\
				4 & -0.3 & -0.3 & 9 & -0.7 & -0.2 \\
				5 & -0.2 & 0.3 & 10 & 0.7 & 0.7 \\
				\hline
			\end{tabular}
		\end{table}
		
		
		\textbf{Answer}
		
		
		\clearpage
		
		
		\subsection{Question 7 (11.3.4)}
		
		For the data presented in Table~11.9, test at the level of significance
		$0.05$ the hypothesis that the regression line is horizontal.
		
		
		\textbf{Answer}
		
		
		\clearpage
		
		
		\subsection{Question 8 (11.3.6)}
		
		For the data presented in Table~11.9, test the hypothesis that when
		$x = 1$, the height of the regression line is $y = 1$ at the level of
		significance $0.01$.
		
		\textbf{Answer}
		
		
		\clearpage
		
		\subsection{Question 9 (11.3.10)}
		
		For the data presented in Table~11.9, construct a confidence interval for
		$\beta_0$ with confidence coefficient $0.95$.
		
		
		\textbf{Answer}
		
		
		\clearpage
		
		
		\subsection{Question 10 (11.3.12)}
		
		For the data presented in Table~11.9, construct a confidence interval for
		$5\beta_0 - \beta_1 + 4$ with confidence coefficient $0.90$.
		
		
		\textbf{Answer}
		
		
		\clearpage
		\subsection{Question 11 (11.3.14)}
		
		For the data presented in Table~11.9, construct a confidence interval with
		confidence coefficient $0.99$ for the height of the regression line at the
		point $x = 0.42$.
		
		
		\textbf{Answer}
		
		
		\clearpage
		
		\subsection{Question 12 (11.3.16)}
		
		Let the statistic $U^2$ be as defined by Eq.\ (11.3.32), and let $\gamma$ be
		a fixed positive constant. Show that for all observed values $(x_i, y_i)$, for
		$i = 1,\ldots,n$, the set of points $(\beta_0^*, \beta_1^*)$ such that
		$U^2 < \gamma$ is the interior of an ellipse in the $\beta_0^* \beta_1^*$-plane.
		
		
		\textbf{Answer}
		
		
		\clearpage
		
		\subsection{Question 13 (11.3.18)}
	\begin{enumerate}
		\item[a.] For the data presented in Table~11.9, sketch a confidence band in the
		$xy$-plane for the regression line with confidence coefficient $0.95$.
		
		\item[b.] On the same graph, sketch the curves which specify the limits at each
		point $x$ of a confidence interval with confidence coefficient $0.95$ for the
		value of the regression line at the point $x$.
	\end{enumerate}
		
		
		\textbf{Answer}
		
		
		\clearpage
		\subsection{Question 14 (11.3.20)}
		
		Suppose that a simple linear regression of miles per gallon $(Y)$ on car
		weight $(X)$ has been performed with $n = 32$ observations. Suppose that the
		least-squares estimates are $\hat{\beta}_0 = 68.17$ and $\hat{\beta}_1 = -1.112$,
		with $\sigma' = 4.281$. Other useful statistics are $\bar{x} = 30.91$, and
		$\sum_{i=1}^n (x_i - \bar{x})^2 = 2054.8$.
		\begin{enumerate}
			\item[a.] Suppose that we want to predict miles per gallon $Y$ for a new observation
			with weight $X = 24$. What would be our prediction?
			
			\item[b.] For the prediction in part (a), find a $95$ percent prediction interval
			for the unobserved $Y$ value.
		\end{enumerate}
		
		
		\textbf{Answer}
		
		
		\clearpage
		
		\subsection{Question 15 (11.3.22)}
		Perform a least-squares regression of the logarithm of the 1980 fish price
		on the 1970 fish price, using the raw data in Table~11.6 on page~707.
		\begin{enumerate}
			\item[a.] Test the null hypothesis that the slope $\beta_1$ is less than $2.0$
			at level $\alpha_0 = 0.01$.
			
			\item[b.] Find a $90$ percent confidence interval for the slope $\beta_1$.
			
			\item[c.] Find a $90$ percent prediction interval for the 1980 price of a species
			that cost $21.4$ in 1970. (Note that $21.4$ is the 1970 price, \emph{not} the
			logarithm of the 1970 price.)
		\end{enumerate}
		
		\textbf{Answer}
		
		
		\clearpage
		
		
	\end{document}
